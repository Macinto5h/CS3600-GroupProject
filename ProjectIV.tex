\documentclass{article}
\usepackage{amsmath,enumitem}
\title{CS3600 Project IV}
\author{Macallan Camara, Peter Simari, and Gary Strzepek}
\begin{document}
  \maketitle
  \par For each of the tables, answer the following questions, and use what you
  have learned in the Normalization chapter to justify your answers:
  \begin{itemize}
  	\item What are the IC's for this table? Turn them over to FDs.
  	\item Where does the table sit in the normalization hierarchy by applying the
  	checking as we went through in class and described in the textbook/notes?
  	\item If this table is not in BCNF, follow the procedure which we just went
  	through in the class, to decompose it into a collection of tables that are in
  	BCNF.
  	\item When you turn the table into BCNF, does it preserve all the FD's? If
  	not, which ones are added and/or dropped?
  \end{itemize}
  \par\textbf{Customer table:}
  \begin{enumerate}[label=\roman*]
  	\item It's primary key is a unique Id that is found in the Person table. \newline
    $FD = Id \rightarrow ReceivesMail$
  	\item BCNF - The customer table is BCNF. There is the Primary key, Id, there are no other non-trivial candidate keys used, therefore there are no non-prime attributes in the table that would be able to be determined through the proper subsets from the non-existant candidate keys. There are also no transitive functional dependencies. There is only one Superkey and it directly determines the other attributes in the table.
  	\item We do not have to do any decomposition since the table is already BCNF.
  	\item
  \end{enumerate}
  \par\textbf{Department table:}
  \begin{enumerate}[label=\roman*]
  	\item It's primary key is the unique attribute Id. \newline
    $FD = Id \rightarrow DeptName$
  	\item BCNF - The department table is similar to the last one. There is the Primary key, Id, there are no other non-trivial candidate keys used, therefore there are no non-prime attributes in the table that would be able to be determined through the proper subsets from the non-existant candidate keys. There are also no transitive functional dependencies. There is only one Superkey and it directly determines the other attributes in the table.
  	\item We do not have to do any decomposition since the table is already BCNF.
  	\item
  \end{enumerate}
  \par\textbf{Employee table:}
  \begin{enumerate}[label=\roman*]
  	\item It's primary key is a unique Id that is found in the Person table. \newline
    $FD = Id \rightarrow \{PayRate, Position\}$
  	\item BCNF - The employee table is similar to the last one. There is the Primary key, Id, there are no other non-trivial candidate keys used, therefore there are no non-prime attributes in the table that would be able to be determined through the proper subsets from the non-existant candidate keys. There are also no transitive functional dependencies. There is only one Superkey and it directly determines the other attributes in the table.
  	\item We do not have to do any decomposition since the table is already BCNF.
  	\item
  \end{enumerate}
  \par\textbf{Person table:}
  \begin{enumerate}[label=\roman*]
  	\item It's primary key is the unique attribute Id. \newline
    $FD = Id \rightarrow \{FirstName, LastName, PhoneNum, Email, Address\}$\newline
    $FD = \{FirstName, LastName\} \rightarrow \{Id, PhoneNum, Email, Address\}$
  	\item BCNF - The Person table is similar to the last one. There is the Primary key, Id, there are no other non-trivial candidate keys used, therefore there are no non-prime attributes in the table that would be able to be determined through the proper subsets from the non-existant candidate keys. There are also no transitive functional dependencies. There is only one Superkey and it directly determines the other attributes in the table.
  	\item We do not have to do any decomposition since the table is already BCNF.
  	\item
  \end{enumerate}
  \par\textbf{Product table:}
  \begin{enumerate}[label=\roman*]
  	\item It's primary key is the unique attribute Id. \newline
    $FD = Id \rightarrow \{Price, Name, Artist, Stock, ReleaseDate\}$\newline
    $FD = \{Artist, Name\} \rightarrow \{Id, Price, Stock, ReleaseDate\}$
  	\item BCNF - The Product table is similar to the department table. There is the Primary key, Id, there are no other non-trivial candidate keys used, therefore there are no non-prime attributes in the table that would be able to be determined through the proper subsets from the non-existant candidate keys. There are also no transitive functional dependencies. There is only one Superkey and it directly determines the other attributes in the table.
  	\item We do not have to do any decomposition since the table is already BCNF.
  	\item
  \end{enumerate}
  \par\textbf{Sales table:}
  \begin{enumerate}[label=\roman*]
  	\item The integrity constraints for the Sales table are that the SalesId and ProductId are the two attributes that make up the primary key. The SalesId alone can uniquely identify the CurrentDate and CustomerId. When the SalesId and ProductId are given
    the following attributes can be found: Quantity, IsReturned, and ReturnDate. When converted to FD's we get the following
    relationships: $SalesId\rightarrow\{CurrentDate,CustomerId\}$, $\{SalesId,ProductId\}\rightarrow Quantity$, and 
    $\{SalesId,ProductId\}\rightarrow\{IsReturned,ReturnDate\}$
  	\item 1NF - The sales table is 1NF because you can easily get the CurrentDate and CustomerId information related to the sale if you were to just use SalesId. Which, violates the 2NF requirement, where you should not be able to determine information from a subset of the candidate key. The candidate key is $\{SalesId,ProductId\}$, and SalesId is a subset from that candidate key.
  	\item
  	\item
  \end{enumerate}
  \par\textbf{WorksIn table:}
  \begin{enumerate}[label=\roman*]
  	\item It's primary key is a unique EmployeeId that is found in the Employee table. \newline
    $FD = EmployeeId \rightarrow \{WorksSince, DeptIt\}$
  	\item BCNF - The WorksIn table is similar to the department table. There is the Primary key, Id, there are no other non-trivial candidate keys used, therefore there are no non-prime attributes in the table that would be able to be determined through the proper subsets from the non-existant candidate keys. There are also no transitive functional dependencies. There is only one Superkey and it directly determines the other attributes in the table.
  	\item We do not have to do any decomposition since the table is already BCNF.
  	\item
  \end{enumerate}
\end{document}
