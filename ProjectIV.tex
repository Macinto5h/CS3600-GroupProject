\documentclass{article}
\usepackage{amsmath,enumitem}
\title{CS3600 Project IV}
\author{Macallan Camara, Peter Simari, and Gary Strzepek}
\begin{document}
  \maketitle
  \par For each of the tables, answer the following questions, and use what you
  have learned in the Normalization chapter to justify your answers:
  \begin{itemize}
  	\item What are the IC's for this table? Turn them over to FDs.
  	\item Where does the table sit in the normalization hierarchy by applying the
  	checking as we went through in class and described in the textbook/notes?
  	\item If this table is not in BCNF, follow the procedure which we just went
  	through in the class, to decompose it into a collection of tables that are in
  	BCNF.
  	\item When you turn the table into BCNF, does it preserve all the FD's? If
  	not, which ones are added and/or dropped?
  \end{itemize}
  \par\textbf{Customer table:}
  \begin{enumerate}[label=\roman*]
  	\item It's primary key is a unique Id that is found in the Person table. \newline
    $FD = Id -> ReceivesMail$
  	\item
  	\item
  	\item
  \end{enumerate}
  \par\textbf{Department table:}
  \begin{enumerate}[label=\roman*]
  	\item It's primary key is the unique attribute Id. \newline
    $FD = Id -> DeptName$
  	\item
  	\item
  	\item
  \end{enumerate}
  \par\textbf{Employee table:}
  \begin{enumerate}[label=\roman*]
  	\item It's primary key is a unique Id that is found in the Person table. \newline
    $FD = Id -> \{PayRate, Position\}$
  	\item
  	\item
  	\item
  \end{enumerate}
  \par\textbf{Person table:}
  \begin{enumerate}[label=\roman*]
  	\item It's primary key is the unique attribute Id. \newline
    $FD = Id -> \{FirstName, LastName, PhoneNum, Email, Address\}$
  	\item
  	\item
  	\item
  \end{enumerate}
  \par\textbf{Product table:}
  \begin{enumerate}[label=\roman*]
  	\item It's primary key is the unique attribute Id. \newline
    $FD = Id -> \{Price, Name, Artist, Stock, ReleaseDate\}$
  	\item
  	\item
  	\item
  \end{enumerate}
  \par\textbf{Sales table:}
  \begin{enumerate}[label=\roman*]
  	\item The integrity constraints for the Sales table are that the SalesId and ProductId are the two attributes that make up
    the primary key. The SalesId alone can uniquely identify the CurrentDate and CustomerId. When the SalesId and ProductId are given
    the following attributes can be found: Quantity, IsReturned, and ReturnDate. When converted to FD's we get the following
    relationships: $SalesId\rightarrow\{CurrentDate,CustomerId\}$, $\{SalesId,ProductId\}\rightarrow Quantity$, and 
    $\{SalesId,ProductId\}\rightarrow\{IsReturned,ReturnDate\}$
  	\item
  	\item
  	\item
  \end{enumerate}
  \par\textbf{WorksIn table:}
  \begin{enumerate}[label=\roman*]
  	\item It's primary key is a unique EmployeeId that is found in the Employee table. \newline
    $FD = EmployeeId -> \{WorksSince, DeptIt\}$
  	\item
  	\item
  	\item
  \end{enumerate}
\end{document}
